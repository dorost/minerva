%%%%%%%%%%%%%%%%%%%%%%%%%%%%%%%%%%%%%%%%%
% University/School Laboratory Report
% LaTeX Template
% Version 3.1 (25/3/14)
%
% This template has been downloaded from:
% http://www.LaTeXTemplates.com
%
% Original author:
% Linux and Unix Users Group at Virginia Tech Wiki 
% (https://vtluug.org/wiki/Example_LaTeX_chem_lab_report)
%
% License:
% CC BY-NC-SA 3.0 (http://creativecommons.org/licenses/by-nc-sa/3.0/)
%
%%%%%%%%%%%%%%%%%%%%%%%%%%%%%%%%%%%%%%%%%

%----------------------------------------------------------------------------------------
%	PACKAGES AND DOCUMENT CONFIGURATIONS
%----------------------------------------------------------------------------------------

\documentclass{article}

\usepackage[version=3]{mhchem} % Package for chemical equation typesetting
\usepackage{siunitx} % Provides the \SI{}{} and \si{} command for typesetting SI units
\usepackage{graphicx} % Required for the inclusion of images
\usepackage{natbib} % Required to change bibliography style to APA
\usepackage{amsmath} % Required for some math elements 

\setlength\parindent{0pt} % Removes all indentation from paragraphs

\renewcommand{\labelenumi}{\alph{enumi}.} % Make numbering in the enumerate environment by letter rather than number (e.g. section 6)

%\usepackage{times} % Uncomment to use the Times New Roman font

%----------------------------------------------------------------------------------------
%	DOCUMENT INFORMATION
%----------------------------------------------------------------------------------------

\title{Building a General Problem Solver from scratch} % Title

\author{Amin \textsc{Dorostanian}} % Author name

\date{\today} % Date for the report

\begin{document}

\maketitle % Insert the title, author and date

Alternative titles:
\begin{enumerate}
\item Bringing (back) AI to mathematics!
\item Theory of everything in Mathematics
\end{enumerate}


% If you wish to include an abstract, uncomment the lines below
% \begin{abstract}
% Abstract text
% \end{abstract}

%----------------------------------------------------------------------------------------
%	SECTION 1
%----------------------------------------------------------------------------------------

\section{Objective}

We aim to build a \textsc{General Pupose Problem Solver} which is able to solve different kind of problems as humans
are able to do. There has been extensive research on applications of Machine Learning, and in a few specific applications machines were able to outperform humans. [give some examples and references]\\

All these advancements in building highly specialized AI powered machines were not successful in building an intelligence nearing human mind. This can be due to several reasons.


Pupose of this framework is to breakdown all concepts and logic into smallest possible parts, to rebuild a whole set of mathematical theorems based on this framework and go beyond.


This project will be the "Super String Theory" (Theory of everything) for Math.
\\

\textbf{Conciousness} was one of the key elements in evolution that made it possible for intelligent beings survive better. But in order to achive our goal we don't necessarily need to give conciousness to an intelligent system. 

The important thing here is to understand the role of conciousness in General Problem Solving ability of humans.
\\

Understanding of a concept or a fact means that human can break it down into parts that is more understanble/imaginable by himself. In this way we actually can say that someone understands mathematical theorem. And in other words being able to answer all Why questions.
\\

We need to show that every problem can be defined as a mathematical problem and there is a solution for that mathematical problem.\\

A mathematical proof can consist several steps to achieve the results. An algorithm also can be defined as a mathematical steps; in this way we can claim that by defining this framework we \textbf{might} be able to build a General Problem Solving framework.\\

[Refer to Godel's Incompleteness theorem here]\\


In order to unify the representation and solve the problem of communicating the knowledge, we will propose a new language to represent concepts.


In order to build an Intelligent Agent that can master it's environment; the agent needs to have two distinct features.

\begin{enumerate}
\item Making experiments and measure the environment, to understand the behaviour of phenomenons.
\item Build solutions to problems based on the facts we already know and map the facts to the environment to understand what is the underlying rule in the environment.
\end{enumerate}


Our focus will be the second part to build and understand all possible fact we can present using already known facts. It is interesting to understand that experiments and measures in the environment can be considered as constraints to the facts we define in our knowledge framework.


It’s so hard to not let our vital needs and unconscious reaction of the mind interfere understanding the true nature of phenomenons happening around us.

Since understanding is not a direct reward for our brain, it’s sometimes even hard for our brain to accept the facts unconsciously regardless of subjective experiences.

\section{Motivation}
The question is why is it important to be able to model a mathematical proof for a theorem?
One of the answers might be the relation between how human brain works to solve different problems.

Being able to reason and build up theorems is the key to understand complicated concepts and apply it in the real world. 

In Mathematics we can define worlds with different constraints, worlds that don't even (but can) exist. There are two important things here to take into account. 


One the most important parts of knowledge is that it needs to be communicatable. Most of the things we experience are subjective and we can understand abstract concepts among these experiences by finding similarites among them. We use various mediums to communicate our knowledge, graphics, symbols and anything can be used to transfer knowledge by making similar experience. Even when we define a concept based on other concepts we understand this by imagining examples of these concepts. Everything is learned by examples.

\begin{enumerate}
\item The knwoledge to find appropriate constraints and representing them in descriptive way.
\item The knowledge to infer/induct a theorem from finite set of principles.
\end{enumerate}


It is hard to understand the concepts before defining the terms appropriately and precisely. So it might be better to redefine some terms from scratch in this new framework.


\textsc{Reality} of all mathematical theorems is that these fact are already there, but since our brain is limited to understand concepts, it is necessary to breakdown facts in smaller facts and explain them accordingly.



\begin{center}\(assumptions \rightarrow statement\)\end{center}

Principles are also assumptions that are \textsc{true} always.



\subsection{Motivation \#2} 

Another motivation can be information overload of subjective understanding, which makes knowledge concepts harder to follow and grasp.



\subsection{Motivation \#3} 

We've been doing lots of efforts to understand what architecture in the brain gives the power of human intelligence. But it is not necessary to understand what is the brain structure, if we understand and rebuild a brain by analyizing behavior of our brain. Problem solving is the key component of intelligence. 


\subsection{Motivation \#4} 
The most important thing in this world is find the correct question. With the right question comes the ultimate solution.


\subsection{Motivation \#5} 

Source of many disputes, wars and genocides might be in the root of subjective understanding of our universe. So if we can build a common ground for all humans and transparently indicate the irrationality of a thought and penalizing based on this metric, we might be able to build a dispute free society.



\section{Background}

There are several attempts about undestanding how we prove mathematical theorems. There are some related attempts that we mention about them. Boolean Satisfiability, SMT and Automated theorem proving.

\begin{enumerate}

\item \textbf{Logic Theorist} was a computer program written by Alan Newell . It was the first program written to mimic human-like problem solving skill. 

\item \textbf{E theorem prover}

\item \textbf{Isabelle}


\item \textbf{Coq}

\end{enumerate}



\section{Terminology}

We are going to redefine concepts and terms used in a new framework that we are going to build.

This framework tells us that all mathematical statemnts that hold true are facts which were there but discovered and understood by us based on breaking them down to principles we can understand with our limited perception.

Propositions, theorems, axioms and priniciples will be redefined in this framework.

\begin{description}
\item[Definitions]
Some concepts are hard to define properly even though they seem obvious among humans. 
\item[Principles]
 
\item[Constraints/Assumptions]

\item[Representation]


\end{description} 
 


\subsection*{Provable Statement Example}

As an example I started analysing Pythagoras theorem which has many different ways of proof which most of them are visual. One of the tricky things to generalize concepts are different representations of concepts. For example all of us know what is a line, we all know what is a triangle; but representing a triangle to show all it's properties is tricky.

%----------------------------------------------------------------------------------------
%	SECTION 2
%----------------------------------------------------------------------------------------



% \begin{figure}[h]
% \begin{center}
% \includegraphics[width=0.65\textwidth]{placeholder} % Include the image placeholder.png
% \caption{Figure caption.}
% \end{center}
% \end{figure}

%----------------------------------------------------------------------------------------
%	BIBLIOGRAPHY
%----------------------------------------------------------------------------------------

% \bibliographystyle{apalike}

% \bibliography{sample}

%----------------------------------------------------------------------------------------


\end{document}